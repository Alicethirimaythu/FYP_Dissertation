% $Id: chapter1.tex 1790 2010-09-28 16:46:40Z jabriffa $

\chapter{Statement of Ethics}

Throughout this project's research, It is my responsibility to consider the legal, ethical, social and professional considerations and ensure to follow basic principles which include:
\begin{itemize}
    \item Do not harm
    \item Informed Consent
    \item Confidentiality of Data
    \item Social Responsibility
\end{itemize}

In the field of machine learning and AI, research projects require the use of datasets which may include personal data in some cases. Such data collection approaches require informed consent from the participants involved. In this project however, no subjects for research are recruited. Since this project did not include any active human participations, collection of informed consent was not a requirement. The datasets used have been collected from secondary sources and are available for public use as is stated by the usage licence (CC0: Public Domain)\cite{Pandey_2021}. Furthermore, there is no personal information such as addresses, user ids from the online platforms and so on included in the datasets. This allows the privacy of individuals that the dataset collected from.

Moreover, the literatures used in the literature review chapter are checked to ensure if the publishers of the papers referred in this project were happy to be used in such manners. This is to ensure to not break any copyright law. All the references of the literature done in the review does not contain word for word paragraphs or sentences and rather includes summarised versions of their completed works and results. All the literature used is cited appropriately not only in the literature review chapter, but throughout the report to make sure that the works done are not plagiarised.

The pre-trained models used in this project are all publicly available, and they have their own responsible use guideline, policy and regulations. These policies are for OpenAI that introduced GPT-2, allow the models to use in commercial such as AI writing assistants, using as capable dialogue agents, translator and speech recognition systems, and research. On the other end, using the model in a malicious manner, including deep fakes, producing misleading new articles or news, impersonate as other online and so on are out of their policies\cite{OpenAI_policy_2019}. For Meta (Llama2), has similar policies and before allowing access to the model, they asked for the authentication through HuggingFace and check if the usage of their model is under their guideline and policies\cite{Meta_guideline_2024}. Other models all have similar guidelines and regulations. These policies ensure the potential misuse of the model to be known as well as how they are trying to lower such misuses, and what they allow the models to use for in public domain. Nonetheless, this project did not involve or use the model in such manners, and only followed their regulations and allowed usages.

Last but not least, this project's purpose is to review different transformer models, both new and old, to see if the improvement of such models improve their abilities tackle downstream task such as emotion classification. Therefore, the researches, implementations, results and evaluation done are for this purpose only. This research project can be referred or further improve to continue meeting the aim of this project however, it does not condone for the misuses or suggesting the models' performance to be able used in a malicious manner. Furthermore, the application of the emotion classification is as mentioned in literature review. This is why this research is done to help tackle the issues.