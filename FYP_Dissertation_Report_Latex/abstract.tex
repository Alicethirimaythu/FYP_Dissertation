% $Id: abstract.tex 1789 2010-09-28 16:30:23Z jabriffa $

\chapter*{Abstract}

The process of extracting emotions out of a piece of text will be of a great importance to better identify and understand the users online in this digital world. As this has many benefits such as getting a sense of the general emotion that the crowd of users leaning towards, especially towards news and events through comments. However, creating machine learning algorithms to accurately classify the emotion of a text is a challenge that
many are facing with any classification. Therefore, this dissertation's aim is to search for better models, mainly transformers models to compare and contrast,
which model works the best. This will be in terms of accuracy, f1-score, confusion matrix and any evaluation functions. The main models that this project will focus on will be BERT based models and Large Language Models, mainly text generative models, as these proposed mechanisms will use multiple self-attention layers. Such that these models could reveal the relationships of each word towards each other and with the emotions present in them, which sparks my motivation to take on this project to further investigate and fine-tune different pre-trained transformers models.