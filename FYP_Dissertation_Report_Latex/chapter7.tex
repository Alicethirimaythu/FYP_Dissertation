% $Id: chapter1.tex 1790 2010-09-28 16:46:40Z jabriffa $

\chapter{Conclusion of this Project}

This chapter is to evaluate if the project follow through the objectives set in the introduction section of this report and conclude the project as a whole. The set objectives are as followed:

\begin{itemize}
    \item Explore different pre-trained transformers models that are both relatively new and old in the fields of NLP.
    \item Review the relevant or similar literatures for emotion classification and the usage of different transformer models.
    \item Discover suitable dataset for training and testing.
    \item Implement the two or more pre-trained models and use the dataset collected to train and test the models implemented.
    \item Provide a critical comparative analysis of the different models used to determine which give the best single-label emotion classification results.
\end{itemize}

The first objective is demonstrated by the implementing 4 different pre-trained transformer models which are MiniLM, RoBERTa, GPT-2 and Llama2. These models are introduced in different timeline in which RoBERTa model is the oldest and the earliest model that is introduced amongst and Llama2 is the recent one which was introduced in 2023.

The relevant literatures for each model are reviewed and discussed in the literature review section with a description of each model and where and how each of them had been used to tackle the emotion classification task. The results of each model from the research are compared and used to support the decision of why these models are implemented in this project.

The dataset used in this project may not be the best dataset, however it is good enough to carry out my research on finding the best suited model for a downstream task such as emotion classification task.

In chapter 3, it has shown the implementations of all 4 models and the usage of the dataset throughout this project. The listings of code, figures and certain output support that all 4 models are implemented. This can be seen further in Test analysis and Evaluation chapter.

Last but not least, this project provides a critical comparative analysis of all 4 models to reach the main aim of this project. This is demonstrated in chapter 4 where training and validation losses are displayed as well as all the overall metrics and confusion matrices are shown to support the critical analysis done in the evaluation chapter. In that chapter, all the models are compared and evaluated with each other as well as with other's models outcome which used the same dataset.

Overall, I believed this project is a success in terms of powering through all the objectives as well as meeting the aim of this project. My motivation to further improve and do more on this emotion classification is still there and are mentioned in my future work.